\documentclass[10pt,table,mathserif]{beamer}
\usetheme[
%%% options passed to the outer theme
%    hidetitle,           % hide the (short) title in the sidebar
%    hideauthor,          % hide the (short) author in the sidebar
%    hideinstitute,       % hide the (short) institute in the bottom of the sidebar
%    shownavsym,          % show the navigation symbols
%    width=2cm,           % width of the sidebar (default is 2 cm)
%    hideothersubsections,% hide all subsections but the subsections in the current section
%    hideallsubsections,  % hide all subsections
%    right                % right of left position of sidebar (default is right)
  ]{Aalborg}

\setbeamertemplate{theorems}[numbered]

\definecolor{watred}{cmyk}{.00,1,1,0.00}
\definecolor{watyellow}{cmyk}{0,0.12,1,0}
\definecolor{watgray}{cmyk}{0,0,0,0.5}

% If you want to change the colors of the various elements in the theme, edit and uncomment the following lines
% Change the bar and sidebar colors:
\setbeamercolor{Aalborg}{fg=black,bg=watred}
\setbeamercolor{sidebar}{bg=white}
% Change the color of the structural elements:
\setbeamercolor{structure}{fg=red}
% Change the frame title text color:
%\setbeamercolor{frametitle}{fg=blue}
% Change the normal text color background:
\setbeamercolor{normal text}{bg=white, fg=black}
\setbeamercolor{alerted text}{bg=white, fg=red}
% ... and you can of course change a lot more - see the beamer user manual.
\usepackage{xcolor}
\usepackage[utf8]{inputenc}
\usepackage[english]{babel}
\usepackage[T1]{fontenc}
\usepackage{threeparttable}
% ... or whatever. Note that the encoding and the font should match. If T1
% does not look nice, try deleting the line with the fontenc.
\usepackage{lmodern}
\usepackage{subfigure}
\usepackage{algorithm}
\usepackage{algorithmic}
\usepackage{colortbl}
\usepackage{biblatex}
\usepackage{bibentry}
\usepackage{epstopdf}
\usepackage{caption}
\usepackage{multirow}

\usepackage{tikz}
\usetikzlibrary{shapes}
\usetikzlibrary{arrows}
\usetikzlibrary{positioning, fit, arrows.meta}
\usepackage{tkz-graph}
\usetikzlibrary{backgrounds,automata}
\bibliography{mybib.bib}



\newcommand*{\Scale}[2][4]{\scalebox{#1}{$#2$}}%
\newcommand*{\Resize}[2]{\resizebox{#1}{!}{$#2$}}%
\newcommand{\vt}[1]{\mathbf{#1}}
\newcommand{\vw}{\mathbf{w}}
%\newcommand{\pm}{\stackrel{+}{-}}
\newcommand{\vx}{\mathbf{x}}
\newcommand{\vi}{\mathbf{i}}
\newcommand{\vo}{\mathbf{o}}
\newcommand{\vxt}{\tilde{\mathbf{x}}}
\newcommand{\vy}{\mathbf{y}}
\newcommand{\impsigma}{\breve{\sigma}}
\newcommand{\barK}{\overline{K}}
\newcommand{\barC}{\overline{C}}
\newcommand{\vz}{\mathbf{z}}
\newcommand{\fnp}{\tilde{f}}
\newcommand{\vu}{\mathbf{u}}
\newcommand{\vs}{\mathbf{s}}
\newcommand{\vc}{\mathbf{c}}
\newcommand{\E}{\mathbf{E}}
\newcommand{\HK}{\mathcal{H}_K}
\newcommand{\XS}{\mathcal{X}}
\newcommand{\DS}{\Delta S}
\newcommand{\Heston}{\textsc{Heston}}
\newcommand{\DVmkt}{\Delta \breve{V}}
\newcommand{\DT}{\Delta_t}
\newcommand{\vuu}{\mathbf{\widetilde u}}
\newcommand{\Real}{\mathbb{R}}
\newcommand{\vdot}[2]{{#1}^T{#2}}
\DeclareMathOperator*{\argmin}{\arg\!\min}
\newcommand{\sym}{\textsc{sym}}
\newcommand{\BS}{\textsc{BS}}
\newcommand{\LOF}{\textsc{lof}}
\newcommand{\svm}{\textsc{svm}}
\newcommand{\AMflag}{\text{mFLAG}}
\newcommand{\rw}{\textsc{rw}}
\newcommand{\diag}{\textsc{diag}}
\newcommand{\sign}{\textsc{sign}}
\newcommand{\MeanAbs}{\E(|\DVmkt-\DS f(\vx)|)}
\newcommand{\Cluster}{\textsc{C}}
\newcommand{\bi}{\text{bi}}
\newcommand{\g}{\mathbf{g}}
\newcommand{\vv}{\mathbf{v}}
\newcommand{\valpha}{\pmb{\alpha}}
\newcommand{\vK}{\pmb{K}}
\newcommand{\vV}{\pmb{\breve{V}}}
\newcommand{\e}{\mathbf{e}}
\newcommand{\vol}{\upsilon}
\newcommand{\vd}{\mathbf{d}}
\newcommand{\vh}{\mathbf{h}}
\newcommand{\vf}{\mathbf{f}}
\newcommand{\vW}{\pmb{W}}
\newcommand{\np}{\text{np}}
\newcommand{\pt}{^{+\Delta t}}
\newcommand{\norm}{\text{norm}}
\newcommand{\row}{\text{row}}
\newcommand{\Vmkt}{\breve{V}}
\newcommand{\vecVmkt}{\mathbf{\breve{V}}}
\newcommand{\Ncut}{\text{Ncut}}
\newcommand{\half}{\frac{1}{2}}
\newcommand{\DKLs}{\bf\textsc{DKL}_{\text{SPL}}}
\newcommand{\DKLg}{\bf\textsc{DKL}_{\text{RBF}}}
\newcommand{\IKLs}{\bf\textsc{IKL}_{\text{SPL}}}
\newcommand{\IKLg}{\bf\textsc{IKL}_{\text{RBF}}}
\newcommand{\LVF}{\textsc{LVF}}
\newcommand{\Del}{\delta^{\textsc{BS}}}
\newcommand{\SABR}{\bf\textsc{SABR}}
\newcommand{\MV}{\bf \textsc{MV}}
\newcommand{\vU}{\pmb{U}}
\newcommand{\vb}{\mathbf{b}}




\nobibliography{Ref.bib}
\definecolor{mycyan}{cmyk}{.2,0,0,0}
\definecolor{mycyan1}{cmyk}{.1,0,0,0}
\definecolor{mycyan3}{cmyk}{.3,0,0,0}
% colored hyperlinks
\newcommand{\chref}[2]{%
  \href{#1}{{\usebeamercolor[bg]{Aalborg}#2}}
}

\title[Data-Driven Models for Discrete
Hedging Problem ]% optional, use only with long paper titles
{Data-Driven Models for   Discrete
Hedging Problem}


\author[Ke Nian ] % optional, use only with lots of authors
{ Ke Nian\\
 Supervisors: Prof.Yuying Li and Prof.Thomas.F.Coleman
}


% - Give the names in the same order as they appear in the paper.
% - Use the \inst{?} command only if the authors have different
%   affiliation. See the beamer manual for an example

%specify some optional logos
\pgfdeclareimage[height=1.4cm]{mainlogo}{logo.png} % placed in the upper left/right corner
\logo{\pgfuseimage{mainlogo}}

\pgfdeclareimage[height=0.75cm]{logo2}{tu-logo} % placed in the lower left/right corner if the \pgfuseimage{logo2} command is uncommented in the \institute command below

\institute[
%  {\pgfuseimage{logo2}}\\ %insert a company or department logo
  David R. Cheriton School of Computer Science, University of Waterloo
] % optional - is placed in the bottom of the sidebar on every slide
{%
  David R. Cheriton School of Computer Science,\\
  University of Waterloo,\\
  Waterloo, Canada
  %there must be an empty line above this line - otherwise some unwanted space is added between the university and the country (I do not know why;( )
}
\date{\today}

\begin{document}
% the titlepage
\begin{frame}[plain] % the plain option removes the sidebar and header from the title page
  \titlepage
\end{frame}
%%%%%%%%%%%%%%%%


\section{Introduction}

\begin{frame}{Local Volatility Model}
The Dupire formula enables us to deduce the volatility function in a local volatility
model from quoted put and call options in the market. 
\begin{itemize}
  \item Under a risk-neutral measure, we assume 
\[
\frac{d S_t}{ S_t}= r(t) dt +\sigma(t,S_t) dZ_t
\]

 \item The forward price for delivery at time T:
\[
F_t = F(t, T) = S_t e^{\int^T_t r(s)ds}
\]
\item We also have:
\[
\frac{d F_t}{ F_t}=\widetilde{\sigma}(t,F_t)dZ_t
\]

\[
\widetilde{\sigma}(t,F_t)=\sigma(t, F_t e^{-\int^T_t r(s)ds})
\]
\end{itemize}
\end{frame}

\begin{frame}{Local Volatility Model}
\begin{itemize}
	\item The forward price of a call option for delivery at time T:	$C(T,K)$
	\item The spot price at t is: $C(T,K) e^{-\int^T_t r(s)ds}$.
	\item It can be shown that, with $\theta(T,K)$ be the normal density function of $S_T$ :
	\[
	C(T,K)=\int_{K}^{\infty} (x-K) \theta(T,x)dx
	\] 
\end{itemize}
\end{frame}

\begin{frame}{Breedon-Litzenberger Formulas}
	If we differentiate pricing function $C(T,K)$ twice we obtain:
	\[
	\frac{\partial C(T,K)}{ \partial K}= \Theta(T,K)-1
	\] 
	\[
	\frac{\partial^2 C(T,K)}{ \partial K^2}= \theta(T,K)
	\] 
\end{frame}

\begin{frame}{Forward Equation}
Let $h$ be an arbitrary function and $v(t,x)=E[h(F_T)|F_t=x]$, we can show that:
\[
E[h(F_T)]=\int_{0}^{\infty} v(t,x) \theta(t,x)dx
\]
Differentiate them with regards to $t$:

\[
0=\int_{0}^{\infty} \frac{\partial v(t,x)}{\partial t} \theta(t,x)dx+\int_{0}^{\infty} \frac{\partial \theta(t,x)}{\partial t} v(t,x)dx
\]
\end{frame}


\begin{frame}{Backward Equation}
Similar to the deduction of Black-Scholes equation, using ito's lemma, we can see that.
\[
\frac{\partial v(t,x)}{\partial t}+\frac{1}{2} \widetilde{\sigma}^2(t,x) x^2 \frac{\partial^2 v(t,x)}{\partial x^2}=0
\]
Note that we are dealing with forward price, the term with interest rate is dropped.
\end{frame}


\begin{frame}{Link Backward Equation and Forward Equation}
Plug in the backward equation:
\[
\frac{\partial v(t,x)}{\partial t}+\frac{1}{2} \widetilde{\sigma}^2(t,x) x^2 \frac{\partial^2 v(t,x)}{\partial x^2}=0
\]
into forward equation:
\[
0=\int_{0}^{\infty} \frac{\partial v(t,x)}{\partial t} \theta(t,x)dx+\int_{0}^{\infty} \frac{\partial \theta(t,x)}{\partial t} v(t,x)dx
\]
We have:
\[
0=-\int_{0}^{\infty}\frac{1}{2} \widetilde{\sigma}^2(t,x) x^2 \frac{\partial^2 v(t,x)}{\partial x^2} \theta(t,x)dx+\int_{0}^{\infty} \frac{\partial \theta(t,x)}{\partial t} v(t,x)dx
\]
\end{frame}



\begin{frame}{Integration By Parts}
Using the rule of Integration By Parts twice:
\[\begin{split}
&\int_{0}^{\infty} \widetilde{\sigma}^2(t,x) x^2 \frac{\partial^2 v(t,x)}{\partial x^2} \theta(t,x)dx\\
=&\widetilde{\sigma}^2(t,x) x^2 \frac{\partial v(t,x)}{\partial x} \theta(t,x)-
\int_{0}^{\infty} \frac{\partial v(t,x)}{\partial x} \frac{\partial [\widetilde{\sigma}^2(t,x) x^2  \theta(t,x)]}{\partial x} dx\\
=&\int_{0}^{\infty}  \frac{\partial^2 [\widetilde{\sigma}^2(t,x) x^2  \theta(t,x)]}{\partial x^2}  v(t,x) dx
\end{split}
\]
Integration By Parts:
\[
\int u v' dx =uv-\int u' v dx
\]
\end{frame}


\begin{frame}{Forward Equation}
Plug in:
\[\begin{split}
&\int_{0}^{\infty} \widetilde{\sigma}^2(t,x) x^2 \frac{\partial^2 v(t,x)}{\partial x^2} \theta(t,x)dx\\
= &\int_{0}^{\infty} \frac{\partial^2 [\widetilde{\sigma}^2(t,x) x^2  \theta(t,x)]}{\partial x^2}  v(t,x)dx
\end{split}
\]
We finally have:
\[\begin{split}
0=&\int_{0}^{\infty} \frac{1}{2} \frac{\partial^2 [\widetilde{\sigma}^2(t,x) x^2  \theta(t,x)]}{\partial x^2}v(t,x) dx-\int_{0}^{\infty} \frac{\partial \theta(t,x)}{\partial t} v(t,x)dx\\
=&\int_{0}^{\infty} [\frac{1}{2} \frac{\partial^2 [\widetilde{\sigma}^2(t,x) x^2  \theta(t,x)]}{\partial x^2} -\frac{\partial \theta(t,x)}{\partial t}] v(t,x)dx
\end{split}
\]
\end{frame}


\begin{frame}{Forward Equation}
Note that, it holds for arbitrary $h(F_t)$ and $v(t,x)$:
\[
0=\int_{0}^{\infty} [\frac{1}{2} \frac{\partial^2 [\widetilde{\sigma}^2(t,x) x^2  \theta(t,x)]}{\partial x^2} -\frac{\partial \theta(t,x)}{\partial t}] v(t,x)dx
\]

We then must have:
\begin{equation}
\frac{\partial \theta(t,x)}{\partial t}=\frac{1}{2} \frac{\partial^2 [\widetilde{\sigma}^2(t,x) x^2  \theta(t,x)]}{\partial x^2}   \label{eq:forward1}
\end{equation}
\end{frame}



\begin{frame}{Dupire’s equation}

\[\begin{split}
&\frac{\partial C(T,K)}{\partial T}=\int_{K}^{\infty} (x-K) \frac{\partial \theta(T,x)}{\partial T} dx\\
= &\int_{K}^{\infty} (x-K) \frac{1}{2} \frac{\partial^2 [\widetilde{\sigma}^2(T,x) x^2  \theta(T,x)]}{\partial x^2} dx\;\;\;  [eq \eqref{eq:forward1}] \\
=& -\frac{1}{2} \int_{K}^{\infty}  \frac{\partial [\widetilde{\sigma}^2(T,x) x^2  \theta(T,x)]}{\partial x} dx\\
=&\frac{1}{2} {\sigma}^2(T,K)K^2  \theta(T,K)\\
=&\frac{1}{2} {\sigma}^2(T,K)K^2  \frac{\partial^2 C(T,K)}{ \partial K^2}
\end{split}
\]
\end{frame}

\begin{frame}{Volatility Interpolation and Option pricing}
Given a time of grid of expiries $0=t_0 < t_1 < \dots $ and a a set of volatility functions, we recursively solve the finite difference  discretization of the Dupure forward equation:
\begin{equation}
\frac{C(t_{i+1},k) -C(t_{i},k)}{t_{i+1}-t_{i}}=\frac{1}{2}\sigma(t_i,k)^2 k^2 \frac{\partial^2 C(t_{i+1},k) }  {\partial k^2} 
\label{eq:dupire}
\end{equation}
\end{frame}
\begin{frame}{Volatility Interpolation and Option pricing}
We can assume $\sigma(t_i,k)$  to be a piecewise constant functions for a given $t_i$ and the initial condition is:
\[C(t_0,k)=max(S(t_0)-k,0)\]
Given the grid of strike: $k_0<k_1<\dots<k_n$ and a expiry $t_i$ and next expiries $t_{i+1}$, let $C_{kk}^j=\frac{\partial^2 C(t_{i+1},k_j) }{\partial k^2}$ and  $C^{j}=  C(t_{i+1},k_j)$. The $C_{kk}^j$ can approximated by finite difference:
\[
C_{kk}^j=\frac{
	\frac{C_{j+1}-C_{j}}{k_{j+1}-k_{j}}-\frac{C_{j}-C_{j-1}}{k_{j}-k_{j-1}}
}{\frac{k_{j+1}-k_{j-1}}{2}}
\]
\end{frame}



\begin{frame}{Volatility Interpolation and Option pricing}
Let us assume $C_{kk}^0=C_{kk}^n=0$. Let 
\[
\alpha_{j}=\frac{1}{(k_{j}-k_{j-1})(k_{j+1}-k_{j-1})}
\]
\[
\beta_{j}=\frac{1}{(k_{j+1}-k_{j})(k_{j}-k_{j-1})}
\]
\[
\eta_{j}=\frac{1}{(k_{j+1}-k_{j})(k_{j+1}-k_{j-1})}
\]
\[
z_j=\sigma(t_i,k_j)^2 k^2 (t_{i+1}-t_{i})
\]

We have:
\[\begin{split}
C(t_{i},k_1)&=-z_1 \alpha_{1}C(t_{i+1},k_1) \\
&+ (1+z_1\beta_{1}) C(t_{i+1},k_2) -z_1\eta_{1} C(t_{i+1},k_2)
\end{split}
\]
\end{frame}



\begin{frame}{Volatility Interpolation and Option pricing}
In matrix form let matrix M be::
\[\tiny
\begin{bmatrix} 
1&  \\
-z_1 \alpha_{1}
& 1+z_1\beta_{1} 
&-z_1\eta_{1}\\
&-z_2 \alpha_{2}
& 1+z_2\beta_{2} 
&-z_2\eta_{2}\\
& & \ddots&\ddots&\ddots \\
&&	&-z_{n-1} \alpha_{n-1}
& 1+z_{n-1}\beta_{n-1}
&-z_{n-1}\eta_{n-1}\\
&&&&&1
\end{bmatrix}
\]
\[M^{-1}\begin{bmatrix}
C(t_{i},k_0)\\
C(t_{i},k_1)\\
C(t_{i},k_2)\\
\vdots\\
C(t_{i},k_{n-1}) \\
C(t_{i},k_{n})
\end{bmatrix}=
\begin{bmatrix}
C(t_{i+1},k_0)\\
C(t_{i+1},k_1)\\
C(t_{i+1},k_2)\\
\vdots\\
C(t_{i+1},k_{n-1}) \\
C(t_{i+1},k_{n})
\end{bmatrix}
\]
\end{frame}


\begin{frame}{Model Calibration}
\[M^{-1}\begin{bmatrix}
C(t_{i},k_0)\\
C(t_{i},k_1)\\
C(t_{i},k_2)\\
\vdots\\
C(t_{i},k_{n-1}) \\
C(t_{i},k_{n})
\end{bmatrix}=
\begin{bmatrix}
C(t_{i+1},k_0)\\
C(t_{i+1},k_1)\\
C(t_{i+1},k_2)\\
\vdots\\
C(t_{i+1},k_{n-1}) \\
C(t_{i+1},k_{n})
\end{bmatrix}
\]
We try to find M that so that $C(t_{i+1},k_j)=C_mkt(t_{i+1},k_j)$. This can be done by:
\[
\inf_{\sigma(t_i,.)} \sum_{j}(\frac{c(t_{i+1},k_j)-c_{mkt}(t_{i+1},k_j)}{Vega_{bs}^{mkt}(t_{i+1},k_j)})^2
\]
\end{frame}
 

\end{document}

