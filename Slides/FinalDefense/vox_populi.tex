\documentclass[10pt,table,mathserif]{beamer}
\usetheme[
%%% options passed to the outer theme
%    hidetitle,           % hide the (short) title in the sidebar
%    hideauthor,          % hide the (short) author in the sidebar
%    hideinstitute,       % hide the (short) institute in the bottom of the sidebar
%    shownavsym,          % show the navigation symbols
%    width=2cm,           % width of the sidebar (default is 2 cm)
%    hideothersubsections,% hide all subsections but the subsections in the current section
%    hideallsubsections,  % hide all subsections
%    right                % right of left position of sidebar (default is right)
  ]{Aalborg}

\setbeamertemplate{theorems}[numbered]

\definecolor{watred}{cmyk}{.00,1,1,0.00}
\definecolor{watyellow}{cmyk}{0,0.12,1,0}
\definecolor{watgray}{cmyk}{0,0,0,0.5}

% If you want to change the colors of the various elements in the theme, edit and uncomment the following lines
% Change the bar and sidebar colors:
\setbeamercolor{Aalborg}{fg=white,bg=watred}
\setbeamercolor{sidebar}{bg=black}
% Change the color of the structural elements:
\setbeamercolor{structure}{fg=watyellow}
% Change the frame title text color:
%\setbeamercolor{frametitle}{fg=blue}
% Change the normal text color background:
\setbeamercolor{normal text}{bg=black, fg=white}
\setbeamercolor{alerted text}{bg=black, fg=watyellow}
% ... and you can of course change a lot more - see the beamer user manual.
\usepackage{xcolor}
\usepackage[utf8]{inputenc}
\usepackage[english]{babel}
\usepackage[T1]{fontenc}
% ... or whatever. Note that the encoding and the font should match. If T1
% does not look nice, try deleting the line with the fontenc.
\usepackage{lmodern} %optional

\usepackage{algorithm}
\usepackage{algorithmic}
\usepackage{colortbl}
\definecolor{mycyan}{cmyk}{.2,0,0,0}
\definecolor{mycyan1}{cmyk}{.1,0,0,0}
\definecolor{mycyan3}{cmyk}{.3,0,0,0}
% colored hyperlinks
\newcommand{\chref}[2]{%
  \href{#1}{{\usebeamercolor[bg]{Aalborg}#2}}
}

\title[Vox Populi: Collecting High-Quality Labels from a Crowd]% optional, use only with long paper titles
{\textsl{Vox Populi}: Collecting High-Quality Labels from a Crowd \\
}


\author[Dekel \& Shamir] % optional, use only with lots of authors
{
  Authors: Ofer Dekel, Ohad Shamir (COLT 2009)\\
  Presenter: Haofan Zhang
}
% - Give the names in the same order as they appear in the paper.
% - Use the \inst{?} command only if the authors have different
%   affiliation. See the beamer manual for an example

%specify some optional logos
\pgfdeclareimage[height=1.4cm]{mainlogo}{logo.png} % placed in the upper left/right corner
\logo{\pgfuseimage{mainlogo}}

\pgfdeclareimage[height=0.75cm]{logo2}{tu-logo} % placed in the lower left/right corner if the \pgfuseimage{logo2} command is uncommented in the \institute command below

\institute[
%  {\pgfuseimage{logo2}}\\ %insert a company or department logo
  David R. Cheriton School of Computer Science, University of Waterloo
] % optional - is placed in the bottom of the sidebar on every slide
{%
  David R. Cheriton School of Computer Science,\\
  University of Waterloo,\\
  Waterloo, Canada
  %there must be an empty line above this line - otherwise some unwanted space is added between the university and the country (I do not know why;( )
}
\date{\today}

\begin{document}
% the titlepage
\begin{frame}[plain] % the plain option removes the sidebar and header from the title page
  \titlepage
\end{frame}
%%%%%%%%%%%%%%%%

% TOC
\begin{frame}{Agenda}{}
\tableofcontents
\end{frame}
%%%%%%%%%%%%%%%%

\section{Introduction}
\subsection{Background}

\begin{frame}{Introduction}{Learning from Crowd}
  \begin{itemize}
    \item Traditional machine learning focuses on the \alert{single-teacher setting} \pause
    \item We are faced with the problem of \alert{learning from crowd} \pause
    \item Therefore, we are interested in identifying and removing low-quality teachers
  \end{itemize}
\end{frame}

\begin{frame}{Introduction}{Learning from Crowd}
  Several challenges: \pause
  \begin{itemize}
    \item No prior knowledge on the identity or the quality of the teacher \pause
    \item No access to gold-set of perfectly labeled examples \pause
      \begin{itemize}
        \item Moreover, a typical teacher only labels a handful set of examples
      \end{itemize} \pause
    \item No control on assignment of examples \pause
      \begin{itemize}
        \item Prevent us from applying \alert{repeated labeling}.
        \item Even applicable, should be avoided because of the cost.
      \end{itemize}
  \end{itemize}
\end{frame}

\begin{frame}{Introduction}{Our Goal}
  Ultimately, our problem is to:
  \begin{itemize}
    \item Work with raw labeled data, with \alert{single noisy label per example}
    \item Detect and eliminate low-quality teaches in a \alert{principled} and \alert{effective} manner
  \end{itemize}
\end{frame}

\begin{frame}{Introduction}{\textsl{Vox Populi}}
  \begin{quote}
    \large
    \alert{Vox populi, vox Dei}\\
    --The voice of the people [is] the voice of God.
  \end{quote}
\end{frame}


\begin{frame}{Introduction}{A Simple Algorithm}
  Suppose we have \alert{multiple labels} for each example \pause
  \begin{itemize}
    \item If most of teaches are good, we can simply take the \alert{average or majority} over repeated labels  \pause
    \item Then we treat this \alert{aggregated label} as ground truth and count incorrect label provided by each teacher \pause
    \item Once we identify low-quality teachers, we can ignore them in the future. \pause
  \end{itemize}
  \begin{center}
  However, we don't have aggregated labels\ldots \\\pause
  \Large We want to \alert{simulate} them!
  \end{center}
\end{frame}

\begin{frame}{Introduction}{A Simple Algorithm}
  Simulating aggregated labels:\pause
  \begin{itemize}
    \item Specifically, we train a hypothesis(classifier) on the entire unfiltered dataset \pause
    \item Then we regard the predictions of this hypothesis as the ground truth. \pause
    \item We \alert{pretend} that we can rely on it, and eliminate low-quality teachers!
  \end{itemize}
\end{frame}


\subsection{Settings and Notations}

\begin{frame}{Settings and Notations}
  We focus on \alert{binary classification} setting:
  \begin{itemize}
    \item Instance space: $\mathcal{X} \subseteq \mathbb{R}^n $ \pause
    \item Test Probability Distribution: $\mathcal{D}: \mathcal{X} \times \{ -1, +1 \} $ \pause
    \item Given dataset: $S = \{ \mathbf{x}_i, y_i \}^m_{i=1}$ \pause
    \item The ML algorithm minimizes:
      \[\hat{F}_{\lambda}(\mathbf{w}, S) = \lambda || \mathbf{w} ||^2 + \frac{1}{m}\sum^m_{i=1}\ell (f(\mathbf{w, x}_i), y_i)\]
    \item Additionally,
      \[f(\mathbf{w, x}_i) = \langle \mathbf{w}, \phi (\mathbf{x}_i) \rangle\]
      represents application of classifier $\mathbf{w}$ to the instance $\mathbf{x}_i$
  \end{itemize}
\end{frame}

\begin{frame}{Settings and Notations}
  In typical supervised learning setting:
  \begin{itemize}
    \item we assume that a training set $S$ is sampled i.i.d from $\mathcal{D}$ \pause
  \end{itemize}
  Here, we introduce an \alert{extra stage} where data is labeled by a set of \alert{$k$} teachers: \pause
  \begin{itemize}
    \item There exists $k$ classifiers \alert{$\{ h_1(\mathbf{x}), h_2(\mathbf{x}),\cdots, h_k(\mathbf{x}) \} : \mathcal{X} \rightarrow \{ -1, +1\}$} \pause
    \item For each unlabeled instance $\mathbf{x}$, we choose a teacher \alert{$t \in \{1,\cdots,k\}$} at random (uniformly here for simplicity) \pause
    \item This results in splitting the sample into $k$ subsets, \alert{$S_1, \cdots, S_k$}
  \end{itemize}
\end{frame}


\begin{frame}{Settings and Notations}
  \begin{itemize}
    \item This process can be viewed as sampling an unlabeled dataset and labeling it using \alert{$\bar{h}(\mathbf{x})$},\\ \pause
          where $\bar{h}(\mathbf{x})$ is the random classifier defined by randomly choosing a hypothesis from $h_1,\cdots,h_k$ \pause
    \item Remeber we want to minimize $\hat{F}_\lambda (\mathbf{w}, S)$:
        \[\hat{F}_{\lambda}(\mathbf{w}, S) = \lambda || \mathbf{w} ||^2 + \frac{1}{m}\sum^m_{i=1}\ell (f(\mathbf{w, x}_i), y_i)\] \pause
    \item Then, it can be seen as the empirical counterpart of minimizing
      \alert{\[F_\lambda (\mathbf{w}) = \lambda ||\mathbf{w}||^2 + \mathbb{E}\left[\ell (f(\mathbf{w,x}), \bar{h}(\mathbf{x}))\right]\]}
    \item We denote \alert{$\mathbf{w}^*$} as the minimizer of $F_{\lambda} (\mathbf{w})$
  \end{itemize}
\end{frame}

\begin{frame}{Settings and Notations}
  \begin{itemize}
    \item Remember that, our goal is to identify and prune away low-quality teachers. \pause
    \item After pruning, only a set of high-quality teachers are left \pause
    \item We denote \alert{$\bar{h}_T(\cdot)$} as randomly pick one classifier from high-quality teachers
  \end{itemize}
\end{frame}

\begin{frame}{Settings and Notations}
  \begin{itemize}
    \item Error rate of teacher $t$:
      \[e_t = \Pr_{(\mathbf{x}, y) \sim D}(y h_t (\mathbf{x})<0)\] \pause
    \item Error rate of entire crowd before pruning:
      \[\bar{e} = \Pr_{(\mathbf{x}, y) \sim D}(y \bar{h} (\mathbf{x})<0)\] \pause
    \item Error rate of entire crowd after pruning:
      \[\bar{e}_T  = \Pr_{(\mathbf{x}, y) \sim D}(y \bar{h}_T (\mathbf{x})<0 | S)\] \pause
  \end{itemize}
  However, we don't know $\mathcal{D}$ nor $h_t$, we cannot calculate $e_t$ directly!\\ \pause
  We need to look at something different!
\end{frame}

\begin{frame}{Settings and Notations}
  The true ``error-rate'' according to $\mathcal{D}$:
  \begin{align*}
    e_t & =  \Pr_{(\mathbf{x}, y) \sim D}(y h_t (\mathbf{x})<0)\\
    \bar{e} & =  \Pr_{(\mathbf{x}, y) \sim D}(y \bar{h} (\mathbf{x})<0)\\
    \bar{e}_T & =  \Pr_{(\mathbf{x}, y) \sim D}(y \bar{h}_T (\mathbf{x})<0 |S)\\
  \end{align*} \pause
  The idea is to look at the ``error-rate'' with respect to $\mathbf{w}^*$:\\
  \begin{align*}
    \epsilon_t & =  \Pr(h_t (\mathbf{x}) f(\mathbf{w^*, x})<0)\\
    \bar{\epsilon} & =  \Pr (\bar{h}(\mathbf{x}) f(\mathbf{w^*, x})<0)\\
    \bar{\epsilon}_T & =  \Pr (\bar{h}_T f(\mathbf{w^*, x})<0 | S)\\
  \end{align*}\vspace{-2ex}
\end{frame}

\begin{frame}{Settings and Notations}
  The idea is to look at the ``error-rate'' with respect to $\mathbf{w}^*$:\\
  \begin{align*}
    \epsilon_t & =  \Pr(h_t (\mathbf{x}) f(\mathbf{w^*, x})<0)\\
    \bar{\epsilon} & =  \Pr (\bar{h}(\mathbf{x}) f(\mathbf{w^*, x})<0)\\
    \bar{\epsilon}_T & =  \Pr (\bar{h}_T f(\mathbf{w^*, x})<0 | S)\\
  \end{align*}\vspace{-2ex}
  Also, it is easy to see:\\
  \[\bar{\epsilon} = \frac{\sum^k_{t=1}\epsilon_t}{k}, \,\,\,\,\, \bar{\epsilon}_T = \frac{\sum^k_{t=1}\mathbf{1}(t \text{ not pruned})\epsilon_t}{|\{t: \text{t not pruned}\}|}\]
  \pause
  So, what is the relationship between $\bar{\epsilon}_T$ and classification error?
\end{frame}

\section{Algorithms and Theories}
\begin{frame}{Algorithms and Theoretical Foundations}{Relating $\bar{\epsilon}_T$ and Classification Error}
  \begin{theorem}
    Assuming $p_D(y|\mathbf{x})\in \{0,1\}$, it holds for any teacher $t$ that
    \[\epsilon_t = \underset{{(\mathbf{x},y)}\sim D}{\Pr}(yf(\mathbf{w}^*, \mathbf{x})<0) + \mathbb{E}_{(\mathbf{x},y)}[e_t(\mathbf{x})\text{sign}(yf(\mathbf{w}^*,\mathbf{x}))]\]
    \label{the1}
  \end{theorem}
\end{frame}

\begin{frame}{Algorithms and Theoretical Foundations}
  \begin{corollary}
    Assume that for any teacher $t$, \alert{$e_t(\mathbf{x})\equiv e_t$} is a constant independent of $\mathbf{x}$. If \alert{$\Pr (sign(f(\mathbf{w}^*, \mathbf{x}))\neq y) < 1/2$},

    then \alert{$\{\epsilon_t\}, \bar{\epsilon}, \bar{\epsilon}_T$} are equivalent to \alert{$\{ e_t \}, \bar{e}, \bar{e}_T$} respectively, up to a uniform, monotonically increasing linear transformation.
    \label{the2}
  \end{corollary}
  \pause
  \begin{itemize}
    \item This means, $\mathbf{w^*}$ does not have to be particularly good, an error-rate smaller than $1/2$ suffices.
  \end{itemize}
\end{frame}


\subsection{Pruning Can Help}

\begin{frame}{Pruning Algorithm}
Motivated by Theorem \ref{the1}, we consider the following simple algorithm to prune teachers: \pause
  \begin{itemize}
    \item Train a classifier $\mathbf{w}'$ on the entire dataset and prune away any teacher for which
      \[\frac{\sum_{i\in S_t}\mathbf{1}(h_t(\mathbf{x}_i)f(\mathbf{w',x}_i)<0)}{|S_t|} > T\]
      for some threshold $T \in (0,1)$ \pause
    \item Essentially, this calculates a rough empirical estimate of $\epsilon_t$, and removes all teachers where this estimate exceeds the threshold $T$ \pause
    \item The question is: Can this actually help???
  \end{itemize}
\end{frame}

\begin{frame}[shrink = 5]{Pruning Can Help}
  \begin{theorem}
    Assume we use the pruning procedure described previously. Also, let $F : [0, 1] \rightarrow [0,1]$ be a cumulative distribution function, such that $F(a) = \frac{1}{k}\sum^k_{t=1}\mathbf{1}(\epsilon_t \leq a)$. Let $P \sim F(\cdot)$, and let $N \sim Poi(m/k)$ be a Poisson random variable with parameter $m/k$. If we assume $m/k = \Theta(1)$ as $m, k$ increase, it holds that
    \[\bar{\epsilon} = \mathbb{E}_P[P]\] and with probability at least $1 - \delta$ over the training sample
    \[ \bar{\epsilon}_T \leq \frac{\mathbb{E}_{P,N}[\Pr (X^P_N \leq NT)P] + r(m, \delta)}{\mathbb{E}_{P,N}[\Pr (X^P_N \leq NT)] - r(m, \delta)}\]
    where $X^P_N$ is a binomial random variable, representing sum of $N$ independent Bernoulli random variables with parameter $P$, and
    \[r(m, \delta) = O \left(\sqrt{\frac{\log (6/\delta)}{m}}\right)\]
    \label{the3}
  \end{theorem}
\end{frame}

\begin{frame}{Pruning Can Help}

\end{frame}

\subsection{Pruning Can't Hurt}

\begin{frame}{Pruning Can't Hurt}
  Another question is:\\
  Can we guarantee that $\bar{\epsilon}_T$ is never considerably larger than $\bar{\epsilon}$? \pause

  \begin{theorem}
    In the setting of Theorem \ref{the3}, it holds for any $\{\epsilon_t\}$ that
    \[\bar{\epsilon}_T \leq \bar{\epsilon} + \frac{2r(m, \delta)}{\mathbb{E}_{P,N}[\Pr (X^P_N \leq NT)]-r(m,\delta)}\]
    where
    \[r(m, \delta) = O \left(\sqrt{\frac{\log (6/\delta)}{m}}\right)\]
    \label{the5}
  \end{theorem}
\end{frame}
\subsection{Setting the Threshold $T$}

\begin{frame}{Setting the Threshold $T$}
  One final question is, how to choose the threshold $T$? \pause
  \begin{corollary}
    In the setting of Theorem \ref{the1}, a sufficient condition for $e_t > \bar{e}$ is
    \[\epsilon_t > \underset{(\mathbf{x},y)\sim D}{\Pr}(y f(\mathbf{w^*,x})<0)+\bar{e}\]
    \label{the6}
  \end{corollary} \pause
  \begin{itemize}
    \item This corollary implies that, if $\epsilon_t$ is larger than a certain quantity, it is definitely worse than average
    \item This suggests a reasonable choice for $T$ is:
        \[\underset{(\mathbf{x},y)\sim D}{\Pr}(y f(\mathbf{w',x})<0)+\bar{e}\]
  \end{itemize}
\end{frame}

\subsection{Reusing the Cleaned Dataset}

\begin{frame}{Reusing the Cleaned Dataset}
  \begin{itemize}
    \item If pruning is successful, we expect to have a cleaner dataset \pause
    \item A more accurate classifier is thereby obtainable \pause
    \item However, the pruning is data-dependent, therefore generalization will be an issue!
  \end{itemize}
\end{frame}

\begin{frame}{Reusing the Cleaned Dataset}
 Fortunately, we can address this easily:
 \begin{itemize}
    \item First, randomly split $S$ into $S_1$ and $S_2$ \pause
    \item Second, we get low-quality teacher set $B_1$ and $B_2$ according to $S_1$ and $S_2$ \pause
    \item Third, use $B_1$ clean $S_2$ to get $S'_2$ and use $B_2$ clean $S_1$ to get $S'_1$ \pause
    \item Finally, train a classifier on $S'$, where
      \[S' = S'_1 \cup S'_2\]
  \end{itemize}
\end{frame}
\section{Experiments}

\subsection{Experiment Settings}
\begin{frame}{Experiments} \pause
  \begin{itemize}
    \item The data-pruning approach is tested using \alert{\textsl{Amazon.com's Mechanical Turk}} \pause
    \item We create an unlabeled set of over \alert{$8,000$} examples, each consists
      \begin{itemize}
        \item A search engine query
        \item An Internet URL \pause
      \end{itemize}
    \item Task was to determine if they are relevant match or not \pause
    \item Each example was labeled by \alert{15} different teachers \pause
    \item A total of \alert{375} individual teachers contributed to the dataset
  \end{itemize}
\end{frame}

\begin{frame}{Experiments}
  \begin{table}
    \centering
    \begin{tabular}{c|c|c|c}
     $\mu$ & $\mu = \infty$(original)  & $\mu = 200$ & $\mu = 50$ \\
     \hline
     No. of Teachers & 375 & 881 & 2509 \\
     Typical Label / Teacher & NA & 14 & 4 \\
    \end{tabular}
    \caption{Description of 3 Datasets}
  \end{table}
  \begin{itemize}
    \item Parameter \alert{$\mu$}: each teacher labels at most \alert{$\mu$} examples \pause
    \item The average of 15 labels are trated as ground truth \pause
    \item The training algorithm is \alert{well-tuned linear SVM}
  \end{itemize}
\end{frame}

\subsection{Experiment Results}
\begin{frame}{Experiments}
 
\end{frame}

\section{Q \& A}
\begin{frame}{Q \& A}
  \LARGE
  \begin{quote}
    \alert{Thank you very much!}\\
    \hspace{8ex} Any Questions?
  \end{quote}
\end{frame}
\end{document}

