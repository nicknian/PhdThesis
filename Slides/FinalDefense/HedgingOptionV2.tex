\documentclass[10pt,table,mathserif]{beamer}
\usetheme[
%%% options passed to the outer theme
%    hidetitle,           % hide the (short) title in the sidebar
%    hideauthor,          % hide the (short) author in the sidebar
%    hideinstitute,       % hide the (short) institute in the bottom of the sidebar
%    shownavsym,          % show the navigation symbols
%    width=2cm,           % width of the sidebar (default is 2 cm)
%    hideothersubsections,% hide all subsections but the subsections in the current section
%    hideallsubsections,  % hide all subsections
%    right                % right of left position of sidebar (default is right)
  ]{Aalborg}

\setbeamertemplate{theorems}[numbered]

\definecolor{watred}{cmyk}{.00,1,1,0.00}
\definecolor{watyellow}{cmyk}{0,0.12,1,0}
\definecolor{watgray}{cmyk}{0,0,0,0.5}

% If you want to change the colors of the various elements in the theme, edit and uncomment the following lines
% Change the bar and sidebar colors:
\setbeamercolor{Aalborg}{fg=black,bg=watred}
\setbeamercolor{sidebar}{bg=white}
% Change the color of the structural elements:
\setbeamercolor{structure}{fg=red}
% Change the frame title text color:
%\setbeamercolor{frametitle}{fg=blue}
% Change the normal text color background:
\setbeamercolor{normal text}{bg=white, fg=black}
\setbeamercolor{alerted text}{bg=white, fg=red}
% ... and you can of course change a lot more - see the beamer user manual.
\usepackage{xcolor}
\usepackage[utf8]{inputenc}
\usepackage[english]{babel}
\usepackage[T1]{fontenc}
\usepackage{threeparttable}
% ... or whatever. Note that the encoding and the font should match. If T1
% does not look nice, try deleting the line with the fontenc.
\usepackage{lmodern} %optional

\usepackage{algorithm}
\usepackage{algorithmic}
\usepackage{colortbl}
\usepackage{biblatex}
\usepackage{bibentry}
\usepackage{epstopdf}
\usepackage{caption}
\usepackage{multirow}






\newcommand*{\Scale}[2][4]{\scalebox{#1}{$#2$}}%
\newcommand*{\Resize}[2]{\resizebox{#1}{!}{$#2$}}%
\newcommand{\vt}[1]{\mathbf{#1}}
\newcommand{\vw}{\mathbf{w}}
%\newcommand{\pm}{\stackrel{+}{-}}
\newcommand{\vx}{\mathbf{x}}
\newcommand{\vxt}{\tilde{\mathbf{x}}}
\newcommand{\vy}{\mathbf{y}}
\newcommand{\impsigma}{\breve{\sigma}}
\newcommand{\barK}{\overline{K}}
\newcommand{\barC}{\overline{C}}
\newcommand{\vz}{\mathbf{z}}
\newcommand{\fnp}{\tilde{f}}
\newcommand{\vu}{\mathbf{u}}

\newcommand{\E}{\mathbf{E}}
\newcommand{\HK}{\mathcal{H}_K}
\newcommand{\XS}{\mathcal{X}}
\newcommand{\DS}{\Delta S}
\newcommand{\Heston}{\textsc{Heston}}
\newcommand{\DVmkt}{\Delta \breve{V}}
\newcommand{\DT}{\Delta_t}
\newcommand{\vuu}{\mathbf{\widetilde u}}
\newcommand{\Real}{\mathbb{R}}
\newcommand{\vdot}[2]{{#1}^T{#2}}
\DeclareMathOperator*{\argmin}{\arg\!\min}
\newcommand{\sym}{\textsc{sym}}
\newcommand{\BS}{\textsc{BS}}
\newcommand{\LOF}{\textsc{lof}}
\newcommand{\svm}{\textsc{svm}}
\newcommand{\AMflag}{\text{mFLAG}}
\newcommand{\rw}{\textsc{rw}}
\newcommand{\diag}{\textsc{diag}}
\newcommand{\sign}{\textsc{sign}}
\newcommand{\MeanAbs}{\E(|\DVmkt-\DS f(\vx)|)}
\newcommand{\Cluster}{\textsc{C}}
\newcommand{\bi}{\text{bi}}
\newcommand{\g}{\mathbf{g}}
\newcommand{\vv}{\mathbf{v}}
\newcommand{\valpha}{\pmb{\alpha}}
\newcommand{\e}{\mathbf{e}}
\newcommand{\vol}{\upsilon}
\newcommand{\vd}{\mathbf{d}}
\newcommand{\vf}{\mathbf{f}}
\newcommand{\np}{\text{np}}
\newcommand{\pt}{^{+\Delta t}}
\newcommand{\norm}{\text{norm}}
\newcommand{\row}{\text{row}}
\newcommand{\Vmkt}{\breve{V}}
\newcommand{\vecVmkt}{\mathbf{\breve{V}}}
\newcommand{\Ncut}{\text{Ncut}}
\newcommand{\half}{\frac{1}{2}}

\newcommand{\DKLs}{\bf\textsc{DKL}_{\text{SPL}}}
\newcommand{\DKLg}{\bf\textsc{DKL}_{\text{RBF}}}
\newcommand{\IKLs}{\bf\textsc{IKL}_{\text{SPL}}}
\newcommand{\IKLg}{\bf\textsc{IKL}_{\text{RBF}}}
\newcommand{\LVF}{\textsc{LVF}}
\newcommand{\Del}{\delta^{\textsc{BS}}}
\newcommand{\SABR}{\bf\textsc{SABR}}
\newcommand{\MV}{\bf \textsc{MV}}





\nobibliography{Ref.bib}
\definecolor{mycyan}{cmyk}{.2,0,0,0}
\definecolor{mycyan1}{cmyk}{.1,0,0,0}
\definecolor{mycyan3}{cmyk}{.3,0,0,0}
% colored hyperlinks
\newcommand{\chref}[2]{%
  \href{#1}{{\usebeamercolor[bg]{Aalborg}#2}}
}

\title[Learning Minimum Variance Discrete Hedging Directly From Market]% optional, use only with long paper titles
{Learning Minimum Variance Discrete Hedging \\Directly From Market}


\author[Ke Nian ] % optional, use only with lots of authors
{ Ke Nian\\
 Supervisors: Prof.Yuying Li and Prof.Thomas.F.Coleman
}


% - Give the names in the same order as they appear in the paper.
% - Use the \inst{?} command only if the authors have different
%   affiliation. See the beamer manual for an example

%specify some optional logos
\pgfdeclareimage[height=1.4cm]{mainlogo}{logo.png} % placed in the upper left/right corner
\logo{\pgfuseimage{mainlogo}}

\pgfdeclareimage[height=0.75cm]{logo2}{tu-logo} % placed in the lower left/right corner if the \pgfuseimage{logo2} command is uncommented in the \institute command below

\institute[
%  {\pgfuseimage{logo2}}\\ %insert a company or department logo
  David R. Cheriton School of Computer Science, University of Waterloo
] % optional - is placed in the bottom of the sidebar on every slide
{%
  David R. Cheriton School of Computer Science,\\
  University of Waterloo,\\
  Waterloo, Canada
  %there must be an empty line above this line - otherwise some unwanted space is added between the university and the country (I do not know why;( )
}
\date{\today}

\begin{document}
% the titlepage
\begin{frame}[plain] % the plain option removes the sidebar and header from the title page
  \titlepage
\end{frame}
%%%%%%%%%%%%%%%%

% TOC
\begin{frame}{Agenda}{}
\tableofcontents
\end{frame}
%%%%%%%%%%%%%%%%

\section{Introduction}


\begin{frame}{Practitioner Black-Scholes (BS) Delta Hedging}

\begin{itemize}
  \item BS model:
\[
\frac{d S}{ S}= \mu dt +\sigma dZ
\]

\[
\sigma:\; \text{Constant}
\]
\item Implied Volatility
  \[
  \sigma_{imp}=V_{BS}^{-1}(V_{mkt},.)
  \]
  \begin{center}
  $V_{mkt}$: market option price \\ $V_{BS}^{-1}$ : inverse of BS pricing function
  \end{center}

\item BS Delta:
\[
\delta_{BS}=\frac{\partial V_{BS}}{ \partial S}
\]
\end{itemize}

\end{frame}






\begin{frame}{Problem with Black-Scholes Delta}
Problem with the traditional Black-Scholes Delta:
\begin{itemize}
  \item Market violates BS assumption
  \item Dependence of volatility on underlying asset price
\end{itemize}
Variants of Hedging Strategy:
\begin{itemize}
  \item Stochastic Volatility Model
  \item Local Volatility Model
  \item Minimum Variance Approach
  \item Indirect Data-Driven Approach
  \item \textbf{Direct Data-Driven Approach}
\end{itemize}
\end{frame}
\section{Delta Hedging Variants}
\subsection{Stochastic Volatility Model}
\begin{frame}{Stochastic Volatility Model}
Stochastic volatility models:
\begin{itemize}
  \item Heston Model
  \item SABR Volatility Model
  \item GARCH Model
\end{itemize}
\end{frame}



\subsection{Minimum Variance Approach}
\begin{frame}{Minimum Variance Approach}
Considering the  the dependence of imply volatility on asset price:
\begin{itemize}
\item The  Minimum Variance (MV) delta:
\[
\delta_{MV}=\frac{\partial V_{BS}}{\partial S}+\frac{\partial V_{BS}}{\partial \sigma_{imp}}\frac{\partial E(\sigma_{imp})}{ \partial S}
\]
\item The authors \footnotemark propose:
\begin{equation}
\frac{\partial E(\sigma_{imp})}{ \partial S}=\frac{a+b\delta_{BS}+c \delta_{BS}^2}{S\sqrt{T}}
\end{equation}
$a, b \text{ and  } c$ are the parameter to be fitted using market data.
\end{itemize}
\footnotetext[1]{Hull,J and White,A ,'Optimal Delta Hedging for Options',
 \\unpublished manuscript}
\end{frame}

\subsection{Local Volatility Model}
\begin{frame}{Local Volatility Model}
The local volatility function (LVF) \footnotemark : volatility is a deterministic function of $S$ and $t$.
 \[
\delta_{MV}=\frac{\partial V_{BS}}{\partial S}+\frac{\partial V_{BS}}{\partial \sigma_{imp}}\frac{\partial E(\sigma_{imp})}{ \partial S}
\]
Local volatility model can also be used to calculate the $\frac{\partial E(\sigma_{imp})}{ \partial S}$

\footnotetext[2]{Coleman, T, Y. Kim, Y. Li and A. Verma,\\ 'Dynamic hedging with a deterministic local volatility function model,' \\Journal of risk, 4 ,1 (2001):63-89}

\end{frame}
\section{Data Driven Approach}
\subsection{Indirect Data-Driven Approach}
\begin{frame}{Problem with Parametric Approach}
Problem with the  previous parametric approaches:
\begin{itemize}
  \item Assumptions do not hold in market
\end{itemize}

Data-driven approach can be more useful in practice.
\begin{itemize}
  \item Minimum assumptions on $S$
  \item Model is determined by market data.
\end{itemize}


\end{frame}


\begin{frame}{Indirect Data-driven Approach}
The indirect data-driven approach \footnotemark can be summarized as following:
\begin{itemize}
\item Let $X$ the features from market.
\begin{itemize}
  \item Asset price $S$
  \item Strike Price $K$
  \item Time to expiration $T-t$
\end{itemize}
\item Determine the data driven pricing function $V(X)$ using regression model.
\item Compute
\[
\delta_{ID}=\frac{ \partial V(X) }{ \partial S}
\]
\end{itemize}
\footnotetext[3]{Hutchinson, James M., Andrew W. Lo, and Tomaso Poggio. "A \\
nonparametric approach to pricing and hedging derivative securities via learning networks." The Journal of Finance 49.3 (1994): 851-889.}
\end{frame}

\subsection{Direct Data-Driven Approach}
\begin{frame}{Problem with Indirect Data-Driven Approach}
Problem with the Indirect Data-Driven Approach:
\begin{itemize}
  \item Intermediate procedure is not necessary.
  \item Not suitable for weekly and monthly hedging
\end{itemize}

Direct data-driven approach can be more useful in practice.
\begin{itemize}
  \item Customized hedging position function
  \item Directly compute the hedging position
\end{itemize}
\end{frame}

\begin{frame}{Direct Data-driven Approach}

The direct data-driven approach is
\[
\min_{f}\left[\frac{1}{N} \sum_{i=1}^N (\Delta V_i-\Delta S_i f(X_i))^2 \right]
\]

\begin{center}
$\Delta V_i$ : the change of option value in data instance $i$\\
$\Delta S_i$ : the change of asset price in data instance $i$ \\
$V_i$ : the option price in data instance $i$ \\
\end{center}
\end{frame}



\section{Model and Experiment}
\subsection{Kernel Learning}
\begin{frame}{Regularized Network}
\begin{itemize}
  \item Indirect data-driven approach:
\[
\min_{f \in RKHS}\left[\frac{1}{N} \sum_{i=1}^N(V_i- f(x_i))^2+\lambda\|f\|^2_K\right]
\]
  \item Direct data-driven approach:
\[
\min_{f \in RKHS}\left[\frac{1}{N} \sum_{i=1}^N(\Delta V_i-\Delta S_i f(x_i))^2+\lambda\|f\|^2_K \right]
\]
\end{itemize}
\end{frame}
\begin{frame}{Regularized Network (2)}
Given
\[
f(x)=\sum_{i=1}^N \alpha_iK(x,x_i)
\]
Indirect data-driven approach:
\[
\min_{\alpha} \; ( K \alpha - V)^T ( K \alpha -V)+ \lambda \alpha^TK\alpha
\]
Direct data-driven approach:
\[
\min_{\alpha} \;( D K \alpha - \Delta V)^T ( D K \alpha -\Delta V)+ \lambda \alpha^T K \alpha
\]

Where $D$ is the diagonal matrix with $\Delta S$ on its diagonal
\end{frame}


\begin{frame}{Regularized Network (3)}
Indirect data-driven approach:
\[
\alpha  =(K+\lambda_P I)^{-1}V
\]

Let $\widetilde{K}=D K$, direct data-driven approach:
\[\begin{split}
&\widetilde{K}^T(\widetilde{K} \alpha -\Delta V)+ \lambda K \alpha =0\\
&(\widetilde{K}^T\widetilde{K}  + \lambda K)\alpha = \widetilde{K}^T \Delta V\\
&\alpha = (\widetilde{K}^T\widetilde{K}  + \lambda K)^{-1} \widetilde{K}^T \Delta V
  \end{split}
\]
\end{frame}

\begin{frame}{Cross-Validation }
For the indirect data-driven approach :
\[
\alpha  =(K+\lambda_P I)^{-1}V
\]
We can calculate the eigen-decomposition of $K=Q \Lambda Q^T$ and then
\[
\alpha  =Q(\Lambda + \lambda I)^{-1} Q^TV
\]
Given the eigen-decomposition, finding $\alpha(\lambda)$  can be done($O(N^2)$).
\end{frame}


\begin{frame}{Leave-One-Out Cross-Validation }
\begin{itemize}
  \item For each data point $x_i$, building a model using the remaining $N-1$ data points, and measuring the error for $x_i$
  \item It can be further shown that, given the eigen-decomposition of $K=Q \Lambda Q^T$, we can estimate the leave-one-out cross-validation errors in $O(N^2)$.
\end{itemize}

\end{frame}


\begin{frame}{Modification of the direct data driven model }
\begin{itemize}
  \item For our direct data-driven approach:
  \[
  \alpha = (\widetilde{K}^T\widetilde{K}  + \lambda K)^{-1} \widetilde{K}^T \Delta V
  \]
  \item If we changed the penalty term from  $\alpha^T K \alpha$ to $\alpha^T  \alpha$:
  \[\alpha = (\widetilde{K}^T\widetilde{K}  + \lambda I)^{-1} \widetilde{K}^T \Delta v\]

\end{itemize}

\end{frame}


\begin{frame}{Modification of the direct data driven model (2) }
Let the SVD of $\widetilde{K}= U \Sigma V^T$,  the $\alpha$ can be determined by

\[
\begin{split}
&(\widetilde{K}^T\widetilde{K}  + \lambda I)\alpha = \widetilde{K}^T \Delta V\\
& ( V \Sigma^T U^T U \Sigma V^T + \lambda I) \alpha =V  \Sigma^T U^T \Delta V \\
&V (  \Sigma^T  \Sigma  + \lambda I) V^T \alpha =V  \Sigma^T U^T \Delta V \\
&( \Sigma^T  \Sigma  + \lambda I) V^T \alpha =  \Sigma^T U^T \Delta V \\
&\alpha = V (\Sigma^T  \Sigma  + \lambda I)^{-1} \Sigma^T U^T \Delta V
\end{split}
\]

$(\Sigma^T\Sigma + \lambda I)$ is again a diagonal matrix. Then we can still estimate the leave-one-out cross-validation errors in $O(N^2)$

\end{frame}

\begin{frame}{Kernels}
In the following experiments, we use:
\begin{itemize}
\item The Gaussian kernel
\[
K(x,y)=e^{-\frac{\|x-y\|_2^2}{2 \sigma_b^2}}
\]
\item The Spline Kernel:
\[
K(x,y)=\int_{lb}^{+\infty} (x-t)_+^d (y-t)_+^d dt + \sum_{k=0}^d x^ky^k
\]
\end{itemize}
\end{frame}

\subsection{Evaluation Criteria}
\begin{frame}{Evaluation Criteria: Local Risk}
The percentage increase in the effectiveness over the BS hedging:
\[
Gain=1-\frac{SSE[\Delta V_i-\Delta S_i\delta^i]}{SSE[\Delta V_i-\Delta S_i\delta^i_{BS}]}
\]
\begin{center}

SSE: sum of squared errors\\
 $\delta$: hedging position computed from different models\\
 $\delta_{BS}$: BS delta\\
 \end{center}
\end{frame}

\subsection{Synthetic Data Experiments}
\begin{frame}{Synthetic Data: Experimental Setting}
\begin{itemize}
\item Dynamic for asset: Heston Model
\item  Training data: a 2 year stock path
\item  Testing Data: 100 independent 6-months stock path
\item  Methods considered:
 \begin{itemize}
 \item  ${\Del}$:  implied  volatility BS delta
 \item $\Heston$:   analytic Heston delta
 \item $\DKLs$ : direct learning a spline kernel hedging  function
 \item $\DKLg$ : directly learning a RBF kernel hedging  function
 \item ${\IKLs}$: Indirectly computing $\frac{\partial f}{\partial S}$  using a spline kernel

\item ${\IKLg}$: Indirectly computing $\frac{\partial f}{\partial S}$ using a RBF kernel

\end{itemize}

\end{itemize}
\end{frame}

\begin{frame}{Daily Hedging Performance}

\begin{table}[htp!]
\small
\begin{center}
\begin{threeparttable}
\begin{tabular}{|l|r c c c c|}
\hline
Method & Gain (\%)& $\MeanAbs$ & Std& VaR & CVaR   \\ \hline
$\Del$ & 0.0 & 0.185 & 0.286 & 0.380 & 0.574 \\
$\IKLg$  & -3.3 & 0.171 & 0.291 & 0.356 & 0.566 \\
$\IKLs$  & -183.3 & 0.291 & 0.482 & 0.669 & 1.105 \\
$\DKLg$  & 63.1 & 0.120 & 0.174 & 0.251 & 0.352 \\
$\DKLs$  & \textbf{64.9} & \textbf{0.121} & \textbf{0.170} & \textbf{0.255} & \textbf{0.345} \\
$\Heston$ & 63.6 & 0.121 & 0.173 & 0.266 & 0.360 \\
%$\DKLgd$  & 62.3 & 0.114 & 0.176 & 0.238 & 0.349 \\
%$\DKLsd$  & \textbf{70.9} &\textbf{ 0.110 }& \textbf{0.154} & \textbf{0.234} & \textbf{0.322} \\
\hline
\end{tabular}
\caption{Daily Hedging Comparison}
\label{Daily}
 \begin{tablenotes}
    \small
  \item[1] FS \#1: $X=\{\textsc{moneyness,~time-to-expiry}\}$
  \item[2] Bold entry indicating best Gain
  \end{tablenotes}
  \end{threeparttable}
\end{center}
\end{table}

\end{frame}










\subsection{Real Data Experiments}
\begin{frame}{Real Data  Hedging Experiments}
\begin{itemize}
  \item Data: S\&P 500 index option from Jan 2007 and Aug 2015
  \item Model Calibration:
    \begin{itemize}
      \item SABR: daily calibration
      \item LVF: $\frac{\partial E(\sigma_{imp})}{ \partial S}$ from implied volatility surface
      \item MV: Use a 36 months time window to train
      \item $\DKLs$: Use a 36 months time window to train. Models are separately calibrated for different Black-Sholes delta range.
    \end{itemize}
  \end{itemize}
\end{frame}



\begin{frame}{S\&P 500 Call Options}
\begin{table}[htp!]
\centering
\begin{threeparttable}
\begin{tabular}{|c |r r r r r|}
\hline
\multirow{3}{*}{Delta}&\multirow{3}{*}{SABR (\%)}&\multirow{3}{*}{\LVF\;(\%)}&\multirow{3}{*}{MV (\%)}&\multicolumn{2}{c|}{$\DKLs$ (\%)}\\
&&&&\multicolumn{2}{c|}{\small Leave-One-Out\tnote{1} }\\
&&&&\multicolumn{1}{c}{\small Traded}&\multicolumn{1}{c|}{\small All}\\ \hline
  0.1 & 42.1 & 39.4 & 42.6 & \textbf{44.1} & \textbf{44.4}  \\
  0.2 & 35.8 & 33.4 & 36.2 & \textbf{37.8} & \textbf{38.1} \\
  0.3 & 31.1 & 29.4 & 30.3 & \textbf{33.1} & \textbf{33.6} \\
  0.4 & 28.5 & 26.3 & 26.7 & \textbf{30.9} & \textbf{31.3}   \\
  0.5 & 27.1 & 24.9 & 25.5 & \textbf{30.0}& \textbf{30.4}  \\
  0.6 & 25.7 & 25.2 & 25.2 & \textbf{29.3}& \textbf{29.8}  \\
  0.7 & 25.4 & 24.7 & 25.8 & \textbf{28.4} & \textbf{30.2}  \\
  0.8 & 24.1 & 23.5 & 25.4&   22.5& \textbf{28.0}  \\
  0.9 & 16.6 & \textbf{17.0} & 16.9 & 8.1  & 12.7  \\
  Overall & 25.7 & 24.6 & 25.5 & \textbf{31.3} & \textbf{26.8}  \\
  \hline
\end{tabular}
\caption{S\&P 500 Call Option Daily Hedging: bold entry indicating best Gain}
\label{SP500Call}
  \begin{tablenotes}
    \small
  \item[1] For each month, the penalties for models are determined by leave-one-out cross validation.
\end{tablenotes}

\end{threeparttable}

\end{table}
\end{frame}


\section{Conclusion and Discussion}
\begin{frame}{Conclusion}
\begin{itemize}
\item loosing assumption on the market dynamic is a good practise.
\item Data driven approach can leads to better performance.
\end{itemize}
\end{frame}

\begin{frame}{Q \& A}
  \LARGE
  \begin{quote}
    \alert{Thank you very much!}\\
    \hspace{8ex} Any Questions?
  \end{quote}
\end{frame}

\end{document}

